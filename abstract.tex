\documentclass[format=acmsmall, review=false, screen=true]{acmart}

\usepackage{booktabs} % For formal tables

\usepackage[utf8]{inputenc}

% Metadata Information
\copyrightyear{2018}
%\acmArticleSeq{9}

% Copyright
%\setcopyright{acmcopyright}
\setcopyright{acmlicensed}
%\setcopyright{rightsretained}
%\setcopyright{usgov}
%\setcopyright{usgovmixed}
%\setcopyright{cagov}
%\setcopyright{cagovmixed}

% Paper history
\received{August 2018}


% Document starts
\begin{document}
% Title portion. Note the short title for running heads
\title{Evolution of Emacs Lisp}

\author{Stefan Monnier}
\affiliation{%
  \institution{Université de Montréal}
  \streetaddress{C.P.\ 6128, succ.\ centre-ville}
  \city{Montréal}
  \state{QC}
  \postcode{H3C 3J7}
  \country{Canada}}
\email{monnier@iro.umontreal.ca}
\author{Michael Sperber}
\affiliation{%
  \institution{Active Group GmbH}
  \streetaddress{Hechinger Str.\ 12/1}
  \city{Tübingen}
  \country{Germany}
}
\email{sperber@deinprogramm.de}


\begin{abstract}
  While Emacs proponents largely agree that it is the world's greatest text
  editor, it is almost as much a Lisp machine disguised as an editor.
  Indeed, one of its chief appeals is that it is \emph{programmable} via
  its own programming language, Emacs Lisp, a Lisp in the classic
  tradition.  Its core has remained remarkably stable since its
  inception in 1981, in large part to preserve compatibility with the many
  third-party packages providing a multitude of extensions.
  Still, Emacs Lisp has evolved and continues to do so.

  Despite the fact that it is closely tied to a concrete editor, Emacs
  Lisp has spawned multiple implementations---in Emacs itself but also
  in variants of Emacs, such as XEmacs and Edwin.  Through competing
  implementations as well as changes in maintainership, it has picked
  up outside influences over the years, most notably from Common Lisp.

  We discuss how important aspects of Emacs Lisp's have been shaped by
  concrete \emph{requirements} of the editor it supports, such as
  the \textit{buffer-local variables} that tie binding to
  editor buffers, as well as implementation constraints.
  These requirements led to the choice of a Lisp dialect as Emacs's
  language in the first place, specifically its simplicity and dynamic nature.
  Loading additional Emacs packages or changing the ones in place
  occurs frequently, and having to restart the editor in order to
  re-compile or re-link the code would be unacceptable.  Fulfilling
  this requirement in a more static language would have been difficult
  at best.

  Emacs Lisp has also picked up some ideas from principled
  programming-language design as well, notably lexical scoping,
  through the involvement of programming-language researchers with
  Emacs maintainership.
  One of Lisp's chief characteristics is its malleability through its
  uniform syntax and the use of macros.  This has allowed the language to
  evolve much more rapidly and substantively than the evolution of its core
  would suggest, by letting Emacs packages provide new
  surface syntax.  In particular, Emacs Lisp can be customized to look
  much like Common Lisp, and additional packages provide multiple-dispatch
  object systems, legible regular expressions, programmable pattern matching
  constructs, generalized variables, and more.
  One area where, maybe surprisingly, Emacs Lisp has not picked up
  ``principled design concepts'' is modularity, despite the fact that
  Emacs constitutes a fairly large ecosystem of independent packages which
  total several million lines of code.  Instead, Emacs Lisp relies on social
  mechanisms to achieve composability of packages.  This, combined
  with Emacs Lisp's dynamic nature, have produced a remarkably robust
  system that continues to function even in the face of unexpected
  exceptions.

  While we highlight these technical aspects of Emacs Lisp as well as the
  underlying motivations for the various design decisions, we trace
  its development chronologically. The timeline of Emacs Lisp
  development is closely tied to the projects and people who have
  shaped it over the years: We document Emacs Lisp history through its
  predecessors, Mocklisp and MacLisp, its early development up the
  "`Emacs schism"' and the fork of Lucid Emacs, the development of
  XEmacs, and the subsequent rennaissance of Emacs development.  When
  it makes sense, we tie in with the evolution of Lisp in general.
\end{abstract}

\ccsdesc{Social and professional topics}
\ccsdesc{Professional topics}
\ccsdesc{History of computing}
\ccsdesc{History of programming languages}

%
% End generated code
%


\keywords{history of programming languages, Lisp, Emacs Lisp}


\maketitle


\end{document}
